\documentclass[11pt]{article}
\usepackage[margin=1.2in]{geometry}
\usepackage{color,soul,hyperref,url}
\usepackage{graphicx,amstext}
\usepackage{subfigure}
\usepackage{amsmath}
\usepackage{booktabs}
\usepackage{listings}
\usepackage{paralist}

\lstset{
  basicstyle=\ttfamily,
  breaklines=true
}

\title{Simulation System Overview}
\author{
  Elaine Wah \\
  \href{mailto:ewah@umich.edu}{ewah@umich.edu}
}
\date{Updated: \today}

\begin{document}

\maketitle

\section{Introduction}

The HFT simulation system employs agent-based modeling and discrete-event simulation to model algorithmic trading in financial markets.

\section{Overview}

The simulation system consists of:
\begin{description}
\item[Markets:] Two types of markets are implemented in the system, a
  continuous double auction (CDA) market and a call market (which matches orders at regular, fixed
  intervals).

\item[Agents:] There are three general types (i.e. roles) of agents in the system,
  which are distinguished by whether or not they have fast access to more than
  one market and whether or not they possess private valuations for the security.

\begin{description}
\item[Background traders:] These agents have (potentially
  undelayed) access to only a single market (which is specified at agent creation as their primary market). 
  % See \href{file:../../src/entity/agent/ZIAgent.java}{\textsf{ZIAgent}}, \href{../src/entity/agent/ZIRAgent.java}{\textsf{ZIRAgent}}, \href{../src/entity/agent/ZIPAgent.java}{\textsf{ZIPAgent}}, anlsd \href{../src/entity/agent/AAAgent.java}{\textsf{AAAgent}}.

\item[Market makers (MM):] These agents submit a ladder of buy and sell orders upon each reentry. 
  % See \href{../src/entity/agent/BasicMarketMaker.java}{\textsf{BasicMM}}, \href{../src/entity/agent/MAMarketMaker.java}{\textsf{MAMM}}, and \href{../src/entity/agent/WMAMarketMaker.java}{\textsf{WMAMM}}.

\item[High-frequency traders (HFT):] These agents have access to
  multiple markets (usually every market). 
  % See \href{../src/entity/agent/LAAgent.java}{\textsf{LAAgent}}.
\end{description}
\end{description}

\section{Discrete-Event Simulation}

In our system, we employ \emph{discrete-event simulation}, a paradigm that
allows the precise specification of event occurrences. It is particularly
effective for modeling communication latencies in the propagation of information between markets and participants.
Components of the simulation system include:

\begin{description}
\item[Entity:] Objects present in the simulation system---e.g., traders, markets, quote processors (QP), transaction processors (TP), and the SIP---that perform actions affecting other entities.

\item[Activity:] Actions that entities can execute.

\item[Event:] A sequence of activities happening at the same time. Maintains the
  order in which they occur.

\item[Event Queue:] Queue ordered by activity time that executes activities in
  the ``proper'' order, sequentially until empty. Multiple activities may occur
  during the same time step, and in most circumstances they
  execute in pseudo-random order according to the random number generator of the
  simulation. The random nature is meant to simulate the slight timing
  differences that occur in real life (nothing actually occurs at the same
  time).
\end{description}

To summarize, an \emph{event} consists of a sequence of \emph{activities} that
are to be executed by various \emph{entities} (traders, markets, and the SIP).


\subsection{Activities}

Each activity has a timestamp and each is associated with at least one entity
present in the simulation system. Note that activities may be chained (the next activity is inserted at the end of the current one).
Activities may also be executely immediately, as each entity has a reference to the event scheduler (responsible for inserting activities). See
Table~\ref{tab:activity} for a list of activities in the system.

\begin{table}
\centering
\begin{tabular}{l l} \toprule
\textbf{Activity} 	& \textbf{Description} \\ \midrule
\textsf{AgentArrival} 	& agent arrives in a market (or markets) \\
\textsf{AgentStrategy} 	& agent executes its trading strategy \\
\textsf{Clear} 		& market clears any matching orders \\
\textsf{Liquidate}	& agent liquidates any net position \\
\textsf{LiquidateAtFundamental}   & agent liquidates inventory at fundamental value \\
\textsf{ProcessQuote} 	& QP (or SIP) updates its best market quotes \\
\textsf{ProcessTransactions}  & TP (or SIP) updates its list of transactions \\
\textsf{SendToQP} 	& market sends a new quote to QP (or SIP) \\
\textsf{SendToTP}   & market sends list of new transactions to TP (or SIP) \\
\textsf{SubmitOrder} 	& agent submits a limit order to a market \\
\textsf{SubmitNMSOrder} 	& agent submits a limit order, routed for best execution \\
\textsf{WithdrawOrder} 	& agent withdraws a specific order from a market \\ \bottomrule
\end{tabular}
\caption{List of activities in the simulation system.}
\label{tab:activity}
\end{table}

\subsection{Example}

To control the latency of the SIP as well as general market access to quote and transaction information, we specify
a set of activities: \textsf{SendToQP}, \textsf{SendToTP}, \textsf{ProcessQuote}, and \textsf{ProcessTransactions}.
%
Two of these activities can be seen in Figure \ref{fig:sim-sys}.  The
\textsf{SendToQP} activity is inserted when a market updates its quote at time
$t$. Once the quote processor (QP) within the SIP gets the information it inserts a
\textsf{ProcessQuote} activity to execute at time $t + \delta$ in the future to
account for the delay caused by processing the information.
%
When \textsf{ProcessQuote} is executed, the QP updates its stored information on
the best market quotes. When agents query the SIP for market information, they
will only get the most recent information that the SIP has processed after $t+\delta$,
not all of the market quotes at the current time.

\begin{figure}
\includegraphics[width=\textwidth]{figs/sip_simulation_example_new}
\caption{Event queue with an example sequence of activities to update the NBBO quote, in which the quote processor in question is the one in the SIP.}
\label{fig:sim-sys}
\end{figure}

\section{Simulation Specification File}

See the \href{simulation_spec.pdf}{simulation spec documentation} for details on specifying the simulation and environment parameters.

\section{Running a Simulation}

\begin{enumerate}
\item To run a basic simulation, create a directory to store the
  simulations (e.g. \path{simulations}) and then create a directory for your
  specific environment configuration inside it:
%
\begin{verbatim}
pwd                 # should print your hft folder
mkdir simulations   # should be already created in the repo
mkdir simulations/test
\end{verbatim}

\item Then copy the default \path{simulation_spec.json} file into the folder you just created:
%
\begin{verbatim}
cp docs/simulation_spec.json simulations/test/
\end{verbatim}

\item Make any tweaks to the specification you want using your favorite text editor.

\item Use the following command to run your simulation:
%
\begin{verbatim}
./run-hft.sh simulations/test [number of simulations to run]
\end{verbatim}

For example:
\begin{verbatim}
./run-hft.sh simulations/test 100
\end{verbatim}

\item The generated observations from 100 simulation runs should be saved within the \path{simulations/test} directory. All log files, if logging is enabled, will be saved in the \path{simulations/test/logs} directory.

\end{enumerate}

Refer to the \href{observations.pdf}{\texttt{observations.pdf}} file for details on interpreting the generated observation files and \href{logging.pdf}{\texttt{logging.pdf}} for reading the log files.

More advanced users can use \verb|run-local-hft.sh| which permits specifying the jar with which you wish to run simulations.

\end{document} 